\documentclass[ident=true,pt,pdftex,11pt,a4paper]{skrapport}
\colortheme{unscathed}%options default, unscathed, cruelwater, violet, skdoc
%opcoes linguagem/dicionario
%\usepackage[portuges]{babel}
\usepackage[utf8]{inputenc}
%\usepackage{identfirst}
%\usepackage[pdftex]{graphicx}

%graph path
\graphicspath{{figures/}}

%graphs
\usepackage{tikz}
\usepackage{pgfplots}
\usepackage{amsmath}

%inicio do documento
\begin{document}
    \begin{titlepage}
        %info. acerca documento trabalho
        \author[ee08001@fe.up.pt]{Ricardo Lopes}
        \date{\today}
        %\logo{uporto-feup.pdf}
        \title{Caracterização estatistica de um Servo-Clock}
        \regarding{Sistemas Distribuidos 2013-2014}
        \maketitle
        \tableofcontent
    \end{titlepage}
    \section{Introdução}
        Nun sistema distribuido o sincronismo entre os relógios dos dispositivo é de enorme relevância.
    \section{Problema}
        A precisão do servo clock é afectada por vários factores, 
    \section{Abordagem}
       Para realizar a caracterização do do atraso do relógio local em relação ao relógio de rederência, foram desenvolvidas duas aplicações, usando o paradigma cliente-servidor
       em que o servo-clock(cliente), periódicamente faz pedidos ao master(servidor) para saber qual é o valor do relógio de referência, A comunicação entre os dois é feita 
       através de mensagens com recurso a sockets TCP/IP.
    \subsection{Master}
        A aplicação implementa basicamente o server socket para a comunicação
    \subsection{Servo-Clock}
    \section{implementação}

     
\end{document}
